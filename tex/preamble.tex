%This is my default preamble, to be inserted right after \documentclass and right before \begin{document}

%Default packages

%Some format things
\usepackage[utf8]{inputenc}
\usepackage[english]{babel}

%AMS math packages
\usepackage{amsmath}
\usepackage{amsfonts}
\usepackage{amsthm}

%Graph plotting package
\usepackage{pgfplots}

%Margins package
\usepackage[margin=0.8in]{geometry}

%More graph plotting
\usepgfplotslibrary{fillbetween}

%Color package
\usepackage{framed}

%New command shortcuts.
\newcommand{\bb}[1]{\mathbb{#1}}
\newcommand{\opr}[1]{\operatorname{#1}}
\newcommand{\veps}[0]{\varepsilon}
\newcommand{\R}[0]{\mathbb{R}}
\newcommand{\Q}[0]{\mathbb{Q}}
\newcommand{\Z}[0]{\mathbb{Z}}
\newcommand{\N}[0]{\mathbb{N}}

%New environments with two different theorem styles.

%Plain theorem style has bolded theorem name, and italicized text
\theoremstyle{plain}
\newtheorem{thm}{Theorem}[subsection]
\newtheorem{lemma}[thm]{Lemma}
\newtheorem{coro}[thm]{Corollary}
\newtheorem{prop}[thm]{Proposition}
\newtheorem{conj}{Conjecture}[subsection]
\newtheorem{ax}{Axiom}[subsection]

%Remark theorem style has italicized theorem name, and default text
\theoremstyle{remark}
\newtheorem{rem}[thm]{Remark}
\newtheorem{definition}{Definition}

%Framing for definitions
\newenvironment{defn}
	{\begin{oframed}\begin{definition}}
	{\end{definition}\end{oframed}}
